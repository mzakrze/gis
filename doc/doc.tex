\documentclass[12pt]{article}%
\usepackage{amsfonts}
\usepackage{fancyhdr}
\usepackage{comment}
\usepackage[a4paper, top=2.5cm, bottom=2.5cm, left=2.2cm, right=2.2cm]%
{geometry}
\usepackage{times}
\usepackage{amsmath}
\usepackage{changepage}
\usepackage{amssymb}
\usepackage{graphicx}%
\usepackage{hyperref}
\usepackage[T1]{fontenc}
\usepackage[polish]{babel}
\usepackage[utf8]{inputenc}

\setcounter{MaxMatrixCols}{30}

\newenvironment{proof}[1][Proof]{\textbf{#1.} }{\ \rule{0.5em}{0.5em}}

\begin{document}

\title{Grafy i sieci\\
    \large SK7 - Izomorficzność drzew nieukorzenionych}
\author{Mariusz Zakrzewski, Szymon Borodziuk}
\date{\today}
\maketitle

\section{Temat pracy}

Tematem pracy jest implementacja i zbadanie dowolnego algorytmu realizującego sprawdzanie izomorficzności drzew nieukorzenionych.

\section{Stos technologiczny}

Wybranym językiem do implementacji programu jest język Scala i/lub Java.

\section{Algorytm}

Dokładny opis algorytmu wybranego do implementacji oraz zbadania znajduje się pod adresem: \url{http://crypto.cs.mcgill.ca/~crepeau/CS250/2004/HW5+.pdf?fbclid=IwAR0qGTkBtEjaB0Icn6KSI0jz5zerCNPDig4h6wiOvu6zpNPSNJoVR3puqsA}

\end{document}